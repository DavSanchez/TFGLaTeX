%\clearpage
\vspace*{\stretch{2}}
\begin{center}
\begin{minipage}{.75\textwidth}
\section*{Presupuesto}
\addcontentsline{toc}{section}{Presupuesto}

En este apartado se hace balance de los costes totales de llevar a cabo este Trabajo Fin de Grado. En el presupuesto se distinguen 3 partidas fundamentales de gasto: recursos materiales, costes de ingeniería y costes en bienes fungibles.
\end{minipage}
\end{center}
\vspace{\stretch{3}}
\clearpage

\subsection*{Coste de recursos materiales}
\addcontentsline{toc}{subsection}{Coste de recursos materiales}

Aquí se detallan los gastos dedicados a los recursos materiales utilizados para desarrollar este TFG. El total de gasto en recursos materiales es de $461.68$\euro. A continuación se detalla cada partida de gasto.

\subsubsection*{Recursos \emph{hardware}}
\addcontentsline{toc}{subsubsection}{Recursos \emph{hardware}}

Considerando una amortización lineal a 3 años de los equipos utilizados durante el tiempo de desarrollo del TFG, en la siguiente tabla se detalla el coste de cada elemento.

 \begin{table}[!ht]
     \begin{center}
     \begin{tabular}{| l  r  r |}
     \hline
     \textbf{Elemento} & \textbf{Coste total (\euro)} & \textbf{Coste imputable (\euro)} \\
     \hline
     Ordenador MacBook Pro 15'' & $1500$ & $333.3$ \\ %\hline
     Tablet Samsung Galaxy GT-P5210 & $150$ & $33.3$ \\ %\hline
     Móvil Sony Xperia L & $350$ & $77.78$ \\ %\hline
     Raspberry Pi 3 y accesorios & $89$ & $17.3$ \\ \hline
\multicolumn{1}{| r }{\textbf{Total}} & & \textbf{$461.68$\euro} \\ \hline
     \end{tabular}
     \end{center}
     %\caption{Coste total del \emph{hardware}}
     \label{costeHard}
     \end{table}%
     
\subsubsection*{Recursos \emph{software}}
\addcontentsline{toc}{subsubsection}{Recursos \emph{software}}

El coste total del \emph{software} son 0\euro, debido a que se ha usado software gratuito o herramientas incluidas con los equipos \emph{hardware} en todo momento. En la siguiente tabla se detalla el \emph{software} utilizado.


\begin{table}[!ht]
\begin{center}
\begin{tabular}{| l  r  r |}
\hline
\textbf{Elemento} & \textbf{Coste total (\euro)} & \textbf{Coste imputable (\euro)} \\
\hline
VS Code (editor de código) & $0$ & $0$ \\ %\hline
Node.js & $0$ & $0$ \\ %\hline
CoovaChilli & $0$ & $0$ \\ %\hline
FreeRADIUS & $0$ & $0$ \\ %\hline
Wireshark (analizador de redes) & $0$ & $0$ \\ %\hline
Sistemas Operativos & $0$ & $0$ \\ %\hline
Navegadores web & $0$ & $0$ \\ %\hline
Paquetes ofimáticos y \LaTeX & $0$ & $0$ \\ %\hline
Otro \emph{software} & $0$ & $0$ \\ \hline
\multicolumn{1}{| r }{\textbf{Total}} & & \textbf{$0$\euro} \\ \hline
\end{tabular}
\end{center}
\label{costeSoft}
\end{table}%

\subsection*{Costes de recursos humanos}
\addcontentsline{toc}{subsection}{Costes de recursos humanos}

Este Trabajo Fin de Grado se ha realizado en un tiempo total de 8 meses a tiempo parcial. Durante este tiempo se han realizado las tareas de estudios previos, desarrollo, pruebas y documentación necesarias para su finalización.

\begin{table}[!ht]
\begin{center}
\begin{tabular}{| l  l r  r |}
\hline
\textbf{Elemento} & \textbf{Meses} & \textbf{Coste mensual (\euro)} & \textbf{Coste final (\euro)} \\
\hline
Estudios previos & $2$ & $0$ & $0$ \\ %\hline
Desarrollo y pruebas & $5$ & $0$ & $0$ \\ %\hline
Documentación & $1$ & $0$ & $0$ \\ \hline
\multicolumn{1}{| r }{\textbf{Total}} & $8$ & & \textbf{$0$\euro} \\ \hline
\end{tabular}
\end{center}
\label{costeHumano}
\end{table}%

\subsection*{Costes de material fungible}
\addcontentsline{toc}{subsection}{Costes de material fungible}

El proceso de impresión, encuadernado y preparación de las tres copias de DVDs (que incluyen la imagen del sistema operativo utilizado, copias de la memoria, y la impresión de la carátula) se realizó íntegramente en una copistería, ascendiendo el coste total a $0$\euro.

\subsection*{Presupuesto total}
\addcontentsline{toc}{subsection}{Presupuesto total}

En la siguiente tabla se resume el coste total del proyecto realizado.

\begin{table}[!ht]
\begin{center}
\begin{tabular}{| l r |}
\hline
\textbf{Elemento} & {Coste total} \\
\hline
Recursos materiales & $0$ \\ %\hline
Recursos humanos & $0$ \\ %\hline
Material fungible & $0$ \\ %\hline
\multicolumn{1}{| r }{Coste previo a impuestos} & $0$ \\ %\hline
\multicolumn{1}{| r }{7\% IGIC} & $0$ \\ \hline
\multicolumn{1}{| r }{\textbf{Total}} & \textbf{$0$\euro} \\ \hline
\end{tabular}
\end{center}
\label{costeTotal}
\end{table}%
\pagebreak
Don José David Sánchez López-Trejo declara que:

El proyecto ``Aplicación Web de sensado colaborativo para obtener acceso gratuito a redes WiFi'', desarrollado como Trabajo Fin de Grado, asciende a un coste total de $0$\euro\,(Aquí va la cantidad en letritas :)).

\indent Firmado: José David Sánchez López-Trejo \\~ \\~ \\~ \\~ \\~
\begin{flushright}
Las Palmas de Gran Canaria, a 19 de noviembre de 2017.
\end{flushright}

\cleardoublepage
\vspace*{\stretch{2}}
\begin{center}
\begin{minipage}{.75\textwidth}
\section*{Pliego de condiciones}
\addcontentsline{toc}{section}{Pliego de condiciones}

En este apartado se exponen de forma detallada las condiciones relativas a la implementación de este sistema.
\end{minipage}
\end{center}
\vspace{\stretch{3}}
\clearpage
\subsection*{Condiciones generales}
Aquí va el pliego de condiciones. (WTF!?)